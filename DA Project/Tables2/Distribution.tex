% Table generated by Excel2LaTeX from sheet 'Sheet7'
\begin{table}[t!]
  \centering
  \caption{Distribution by DA Works and State}  \label{tab:distribution}%
  \resizebox{\textwidth}{!}{
  \begin{threeparttable}
    \begin{tabular}{lHcHcccccc}
    \toprule
          & \multicolumn{9}{c}{State} \\
\cmidrule{2-10}          & ACT   & NSW   & NT    & QLD   & SA    & TAS   & VIC   & WA    & Total \\
    \midrule
    Non-DA & 13,421 & 245,948 & 12,707 & 319,273 & 77,504 & 35,803 & 255,412 & 129,746 & 1,063,686 \\ % 1,089,814 \\
    Carports/Garages/Sheds &       & 301   &       & 307   & 98    & 23    & 194   & 149   & 1,072 \\
    Duplex &       & 612   &       &       & 1     &       & 122   &       & 735 \\
    Extension/Alteration &       & 4,616 &       & 681   & 726   & 74    & 3,412 & 1,208 & 10,717 \\
    House/Single Dwelling &       & 1,450 &       & 584   & 514   & 15    & 2,676 & 783   & 6,022 \\
    Multiple DA &       & 1,239 &       & 1,194 & 763   & 10    & 1,528 & 4,324 & 9,058 \\
    Swimming Pool &       & 3,891 &       & 3,626 & 1,200 & 8     & 2,640 & 4,966 & 16,331 \\
    Verandahs \& Pergolas &       & 1,072 &       & 1,976 & 1,702 & 103   & 2,746 & 4,199 & 11,798 \\
    Total & 13,421 & 259,129 & 12,707 & 327,641 & 82,508 & 36,036 & 268,730 & 145,375 & 1,119,419 \\ % 1,145,547 \\
    \bottomrule \\ [-2ex] 
\end{tabular}%
  \begin{tablenotes}[para,flushleft]
  \LARGE
     % The table shows the distribution of DAs by type and state. We see that there are a total of around 1.08m unimproved homes. For improved homes, the highest number of DAs in our sample are of the type Swimming Pools - 16,331, followed by Verandas/Pergolas - 11,798 and Extension/Alterations - 10,717. The least number of DAs in our sample are for Duplexes - 735 and Carport/garages/sheds - 1,072. Across states, we have highest number of DAs in VIC - 13,196 followed by NSW -13,181. ACT and NT do not have any observations for DAs.
\end{tablenotes}    

    \end{threeparttable}
    }

  \bigskip

  \caption{Summary Statistics - DA Cost (Constant Dollar 2016-Q3)} \label{tab:DA_cost_statistics_rs_const_dollar}%
   \resizebox{\textwidth}{!}{
    \begin{threeparttable}
    \begin{tabular}{rccccccc}
    \toprule
          & \multicolumn{7}{c}{Mean} \\
          & \multicolumn{7}{c}{(Std. deviation)} \\
\cmidrule{2-8}    \multicolumn{1}{c}{DA Type} & \multicolumn{1}{c}{NSW} & \multicolumn{1}{c}{QLD} & \multicolumn{1}{c}{SA} & \multicolumn{1}{c}{TAS} & \multicolumn{1}{c}{VIC} & \multicolumn{1}{c}{WA} & \multicolumn{1}{c}{All States} \\
    \midrule
    \multicolumn{1}{l}{Carports/Garages/Sheds} & 17,136 & 14,935 & 11,061 & 15,296 & 14,279 & 14,586 & 15,039 \\
          & (12,799) & (10,527) & (6,329) & (8,745) & (12,468) & (8,717) & (11,139) \\
    \multicolumn{1}{l}{Duplex} & 491,765 &       & 172,374 &       & 760,411 &       & 535,922 \\
          & (233,881) &       & NA    &       & (278,275) &       & (261,690) \\
    \multicolumn{1}{l}{Extension/Alteration} & 107,353 & 65,979 & 85,077 & 82,035 & 138,539 & 129,736 & 115,492 \\
          & (106,108) & (74,837) & (69,411) & (84,990) & (143,566) & (140,012) & (121,634) \\
    \multicolumn{1}{l}{House/Single Dwelling} & 295,850 & 288,558 & 240,484 & 239,705 & 377,409 & 389,903 & 338,749 \\
          & (188,719) & (115,932) & (128,832) & (128,905) & (282,821) & (285,465) & (244,922) \\
    \multicolumn{1}{l}{Multiple DA} & 282,546 & 251,220 & 208,279 & 153,836 & 375,217 & 274,928 & 284,015 \\
          & (272,418) & (217,374) & (146,280) & (104,467) & (297,933) & (229,289) & (245,612) \\
    \multicolumn{1}{l}{Swimming Pool} & 27,846 & 29,349 & 27,320 & 38,776 & 40,657 & 24,360 & 29,157 \\
          & (16,029) & (13,238) & (18,725) & (36,661) & (32,755) & (100,236) & (58,170) \\
    \multicolumn{1}{l}{Verandas/Pergolas} & 20,188 & 15,309 & 12,810 & 18,859 & 16,507 & 10,907 & 14,135 \\
          & (15,041) & (12,909) & (8,954) & (22,191) & (14,186) & (8,018) & (11,958) \\
    \multicolumn{1}{l}{All DA Types} & 129,786 & 78,234 & 79,963 & 59,266 & 173,012 & 116,434 & 123,863 \\
          & (182,441) & (133,177) & (115,008) & (86,891) & (240,681) & (199,600) & (194,603) \\
    \bottomrule \\ [-2ex] 
    \end{tabular}%
   
   \begin{tablenotes}[para,flushleft]
  \LARGE
     %This table shows improvement spending by type and state reported in constant dollar of 2016-Q3. We see that Duplex has the highest average cost of development of \$535,922 followed by House/Single Dwelling with average cost of \$338,749. Developing a House/Single Dwelling is most expensive in Western Australia. This is because the houses in Western Australia are bigger and also remote which adds to increased transportation costs. Verandas/Pergolas have the least average cost of \$14,135 and then followed by Carports/Garages/Sheds with average cost of around \$15,039. Swimming Pools in Victoria are most expensive to built with average cost of \$40,657 due to the heating systems in Victorian pools. Across states, we see that the highest improvement spending is done in Victoria with average spending of \$173,012 followed by NSW with average spending of \$129,786.
\end{tablenotes}   
    
 \end{threeparttable}  
    
    }%
   
 %\bigskip 

\end{table}%
