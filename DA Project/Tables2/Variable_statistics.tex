% Table generated by Excel2LaTeX from sheet '9 june 2017'
\begin{table}[!p]
  \centering
  \caption{Variable Descriptions and Sample Summary Statistics}
  \resizebox{0.93\textwidth}{!}{

  
    \begin{tabu} to \textwidth {X[2,l]X[3,l]X[0.6,c]X[0.6,c]X[0.6,c]}
    \toprule
          &       & \multicolumn{1}{c}{Overall} & \multicolumn{1}{c}{Improved} & \multicolumn{1}{c}{Unimproved} \\
\cmidrule{3-5}    Name  & Description & \multicolumn{1}{c}{Mean} & \multicolumn{1}{c}{Mean} & \multicolumn{1}{c}{Mean} \\
   &    &   \multicolumn{1}{c}{(Std. Deviation)} & \multicolumn{1}{c}{(Std. Deviation)} & \multicolumn{1}{c}{(Std. Deviation)} \\
    \midrule
     
     Number of observations &       & \multicolumn{1}{c}{1,145,547} & \multicolumn{1}{c}{55,733} & \multicolumn{1}{c}{1,089,814} \\
    &&&&\\
    
    Purchase Price & House price at first sale & \multicolumn{1}{c}{380,867} & \multicolumn{1}{c}{436,829} & \multicolumn{1}{c}{378,005} \\
          &       & \multicolumn{1}{c}{(319,547)} & \multicolumn{1}{c}{(410,073)} & \multicolumn{1}{c}{(313,949)} \\
   Expected Purchase Price at DA  &   Purchase Price indexed to DA (DA homes)    &       & \multicolumn{1}{c}{554,532} &  \\
          &       &       & \multicolumn{1}{c}{(478,355)} &  \\
    DA cost & Cost of developments as reported in development applications &       & \multicolumn{1}{c}{104,345} &  \\
          &       &       & \multicolumn{1}{c}{(165,288)} &  \\
    Purchase price (notional) & Price at purchase for unimproved homes; Indexed Purchase Price plus cost of development for DA homes & \multicolumn{1}{c}{391,670} & \multicolumn{1}{c}{658,879} & \multicolumn{1}{c}{378,005} \\
          &       & \multicolumn{1}{c}{(333,766)} & \multicolumn{1}{c}{(536,042)} & \multicolumn{1}{c}{(313,949)} \\
    Resale Price & House price at second sale & \multicolumn{1}{c}{506,030} & \multicolumn{1}{c}{887,089} & \multicolumn{1}{c}{486,542} \\
          &       & \multicolumn{1}{c}{(411,784)} & \multicolumn{1}{c}{(723,141)} & \multicolumn{1}{c}{(379,065)} \\
    SSD specific hedonic index growth rate & Annual average growth rate of statistical sub-division specific hedonic Index for property type 'house' for period 1990 - 2016 & \multicolumn{1}{c}{0.073} &       &  \\
    Log Returns & log(Resale price / Purchase price (notional)) & \multicolumn{1}{c}{0.279} & \multicolumn{1}{c}{0.304} & \multicolumn{1}{c}{0.277} \\
          &       & \multicolumn{1}{c}{(0.409)} & \multicolumn{1}{c}{(0.422)} & \multicolumn{1}{c}{(0.408)} \\
    Months between sales (notional) & No of months between sales for non modified homes; No. of months between notional purchase and resale for DA homes  &   \multicolumn{1}{c}{50.2}   & \multicolumn{1}{c}{47.7} & \multicolumn{1}{c}{50.4} \\
    
       &       & \multicolumn{1}{c}{(31.5)} & \multicolumn{1}{c}{(30.9)} & \multicolumn{1}{c}{(31.5)}\\
    
  %  DA Types & Duplex, Extension/Alteration, Garages/Sheds \& Carports,House/Single Dwelling, Multiple DA, Swimming Pool and Verandahs \& Pergolas &       &       &  \\
    \bottomrule \\[-1.8ex]
    
     \multicolumn{5}{l}{\parbox{19cm}{\LARGE This table shows the summary statistics for the main variables, directly or indirectly, used in our model. We have a total of 1.1m records used in our model with 55,733 improved homes as our treatment sample and 1.08m unimproved homes as our control sample. The average purchase price for improved and unimproved homes is \$436,829 with a standard deviation of \$410,073 and \$378,005 with a standard deviation of \$313,949. The expected purchase price at the time of DA, for improved homes is \$554,532 and with average improvement cost of \$104,345, the average notional purchase price is \$658,879 with a standard deviation of \$536,042. The notional purchase price for unimproved homes is same as the purchase price. The average resale price for improved and unimproved homes is \$887,089 with standard deviation of \$723,141 and \$486,542 with standard deviation of \$379,065 respectively. The large standard deviation of improved homes is mainly due to the different types of home improvements as some improvements for e.g. verandas/pergolas add little value while others e.g. Duplex adds more value to the homes. The average annual growth rate of Statistical Subdivision level Hedonic House Price Index used to calculate the expected purchase price and also the market return is 7.3\%. We see that the average log return of the resale price to the notional purchase price (i.e. after accounting for the cost of improvement) for improved homes is 30.4\% with a standard deviation of 42.2\%. While for unimproved homes the average log return is 27.7\% with a standard deviation of 40.8\%. In terms of univariate results, this may suggest that the return for improved homes is, on average, 2.7\% higher than unimproved homes. However, the univariate results do not control for factors such as time of purchase and resale, depreciation of the home, and property attributes such as number of beds, baths and cars which is possible in the regression model. After controlling for these factors in the regression, we find that the returns for improved homes are in fact lower, on average, than unimproved homes by 2.4\%. The average time between sales for unimproved homes is 47.7 months while for improved homes the average time between DA and resale (i.e. the time between notional purchase and resale) is 50.4 months. These are the times for which the households have consumed the housing service and allows for capturing the depreciation effect of the property.}}
    
    % the figures reported in unimproved home are for 1.08m are post 2004, but the need to be checked for constant dollars of 2016..and then update the figures above in notes..
    
    \end{tabu} %
        }
 
  \label{tab:variable_statistics}%
  
\end{table}%
