

some articiles
http://www.sciencedirect.com/science/article/pii/S0094119099921587
http://www.sciencedirect.com/science/article/pii/S0094119010000112
http://www.sciencedirect.com/science/article/pii/S0094119003000810
http://www.sciencedirect.com/science/article/pii/009411909290022D
http://www.sciencedirect.com/science/article/pii/S0094119006000763
http://www.sciencedirect.com/science/article/pii/0094119077900067 Empirical evidence on home improvements
http://www.sciencedirect.com/science/article/pii/S1051137796900194 Depreciation, Maintenance, and Housing Prices



construction times home pages..
2 months for bathroom
1.5 months for kitchen




Renovations are a way of achieving desired housing outcomes without moving. People may choose to renovate (rather than move) because they like their house or location, because it is cheaper than moving or to add value to their home.1,2 In addition, upgrading older housing stock after purchase may be a way of entering the housing market in more expensive, established areas. In the last two decades of the 20th century, renovating - whether adding or changing a feature, restoring or extending the house - made an increasingly important contribution to the building industry, which suggests a growing trend towards home renovation.3

In 1999, 58% of owner occupiers stated that some renovations had been carried out on their current dwelling in the previous 10 years (a total of 2.9 million dwellings), while 27% stated that renovations had been carried out on the dwelling in the previous two years (1.3 million dwellings). The relatively high proportion of dwellings renovated in the previous two years reflects in part the beginning of a boom in the building industry prior to the introduction of the GST in July 2000.4 It may also reflect an increase in the popularity of renovating, the likelihood for an individual dwelling to undergo multiple renovations over time, or people’s lower awareness or recollection of renovations which have occurred over a longer period.

This article is restricted to owner occupiers. While renovations are conducted on the dwellings of renter households, they were less common (29% of renter households reported that they lived in renovated dwellings, compared with 58% of owner occupiers). Because they tend to be carried out and paid for by the property owner, the current tenants are less likely to be aware of the costs or nature of renovations carried out either prior to or during their tenancy. 


Renovating by owner occupiers
The main source of data for this article is the ABS 1999 Australian Housing Survey (AHS) conducted between September and December 1999, which collected information on renovations carried out on residential dwellings in the previous two and 10 years, as well as other household and dwelling characteristics (see Australian Housing Survey: Housing Characteristics, Costs and Conditions, 1999, ABS cat. no. 4182.0). 

Renovations include alterations and additions, but exclude repairs and maintenance to the dwelling (see Australian Social Trends 2002, Housing condition and maintenance). This article refers to renovations which were carried out (to the knowledge of the current residents) in the two years preceding the survey (1997-99) and the 10 years preceding the survey (1989-99). These renovations may or may not have been carried out by the current residents.

This article is restricted to owner occupiers - households whose dwelling was owned by one or more resident, either with or without a mortgage.

Renovated dwellings (homes) are owner-occupied dwellings which in 1999 had, to the owner’s knowledge, undergone some type of renovation(s) in the previous 10 years.

The reference person for each household is chosen using the following criteria, in order of precedence:
the person with the highest tenure type ranked in order of owner without a mortgage, owner with a mortgage, renter and other tenure;
the person with the highest income; and
the eldest person.


People living in renovated homes
In 1999, 62% of family households (compared with 50% of non-family households) lived in dwellings which had undergone some type of renovation in the previous 10 years. Of all family households, couples with children were most likely to live in a renovated home (67%). 

Higher income households were more likely to live in renovated dwellings than lower income households (65% of households in the top two income quintiles, compared with 47% in the lowest two). This is consistent with higher income households being more able to afford either renovations or the purchase of recently renovated (and therefore more expensive) dwellings.

In 1999, approximately 63% of all households where the reference person was of workforce age (15-64 years) lived in a dwelling which had been renovated in the last 10 years. Households with a reference person aged 25-44 years were the most likely to live in a renovated dwelling (66%), while those with a reference person aged 65 years or over (retirement age) were the least likely to live in a renovated dwelling (43%). The higher propensity for households with a reference person of workforce age to live in a renovated dwelling in 1999 is linked with the higher incomes of households living in renovated dwellings, as household income is usually highest when its members are 



Report descriptive statistics of the variables 
Min max mean median count 



Distribution of returns in annual yields for with and without da's
Distribution of da values

Distribution of prev sale and da value and cur sale as of cur sale. 
    

Look at the distribution of time between the da's and repeat sales before and after. As in distribution of how long someone takes to do da after purchase. And sfter how long on average someone  sells the property. 
    This is done...
    

Show the line of shifted time scale distribution of Da's





Data transformation 
Ssd index monthly level
Cpi index quarterly uniform across Australia

Models
Max value comparison 
Robust linear regression for outliers-- 
        This model was not tried
Try using builder name model ---- 
        You: this had 139k obs for da repeat sales however the model was scrap.. with no fit poor rsq..

Try log models --- 
        
Track changes is off
You: the did model is in log prices...
Mar 10, 2017 4:43 PM • Edit
Try level models -- 
    You: the markup model is in levels -- the DID model is in log prices..

Try age of da variable in the model --- 
        You: this is good.. done the variable is significant in the model and inlcuded..

Understand interaction effects in the model --
    You: this is good.. done ..included the interaction effects in the model.

Try suburb as factors. 
    You: tried suburb level model in did and the system crashed due to 116 gb memory requiremnt..

Try vacant land - 
    You: done excluded from the data



-----Vito Notes---

1.     

Table that describes your da data and the

filters that applied to obtain final data set


a.     

Identify number, type and descriptives around da





b.     

Identify number of properties affected, you may

want to consider including descrptives around the averaage growth rates for

such properties





c.     

You may consider putting in a figure to explain

your methodlody for markup and da method





2.     

Table that describes sales and return  inforamtion





a.     

Sepearte sample in DA affected and non da

affected





b.     

Explain the distribution of proeprty sales and

time to da and time to subsequent sale





3.     

Plot the index for each state/city you have in

yoru data, this is relevant to explain the markup





4.     

Table the results of your regression of markup





a.     

Panel A, models 1-8





b.     

Panel b, interaction for type of DA, but just

use model





5.     

Plot the repeat sales index for da and non da

homes, this could be your brideg to test whether DA homes are different from

nonDA home that res-sell





a.     

You will need to include a footnote stating the

method you followed, I don’t care about CS or Calhoun, I want 2 lines





b.     

Use the adjusted for cost model





6.     

DID result




18 aprill...
tried to run the model ratio of current to previous beds baths cars .. to make the model more parsimonous.. however the coefficient are negative and that is not interpretable..
also the number of observations get reduced as there is problem of infinity due to divisor beign 0.. so it is probably better to keep the model as current and previous beds baths as separate variables..

since resolution of this question was beyong the scope of our analysis, we experimented with both the cases..


put this as footnote.. we also experimented with semi log, log-log and standardised models..



\cite{boehm1986improvement}  The author also assumes that the marginal cost curve is constant and that the cost of the home improvement grows linearly with the quantity of the home improvement demanded.
the physical quantity of improvement can augment to the initial level of housing stock capital and by reducing the physical depreciation of housing capital

\cite{boehm1986improvement} find that despite adding adding serveral external variable only a fifth of the variance in home improvemnt is expalned by their model. Others also have limited success in explaining the home improvement expenditures..  They expain that one possible reason for low explanatory power of home improvement models is that these expenditures may have a large stochastic component.. They suggest that in many cases what homeowner decide to do with their properties has little to do with measurable variables..

In our model, we our able to explain more than the third of the variances. a lot of explanatory power is lost as we are modelling the value of the housing service over and above the expected the selling price. In addition there is spatial heterogeneity which is only accounted for at state level and any variances below state level is uncontrolled..


as a foonote -- we focus on the consumption demand i.e. owner occupiers. Investment demand is beyond the scope of this study..

we also find that our markup was 0.04\% and was relatively inelastic..